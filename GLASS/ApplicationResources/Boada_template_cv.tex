\documentclass[margin,line, 11pt]{Boada_res}

\oddsidemargin -.5in
\evensidemargin -.5in
\textwidth=6.0in
\itemsep=0in
\parsep=0in

% if using pdflatex:
\setlength{\pdfpagewidth}{\paperwidth}
\setlength{\pdfpageheight}{\paperheight} 

\newenvironment{list1}{
  \begin{list}{\ding{113}}{%
      \setlength{\itemsep}{0in}
      \setlength{\parsep}{0in} \setlength{\parskip}{0in}
      \setlength{\topsep}{0in} \setlength{\partopsep}{0in} 
      \setlength{\leftmargin}{0.17in}}}{\end{list}}
\newenvironment{list2}{
  \begin{list}{$\bullet$}{%
      \setlength{\itemsep}{0in}
      \setlength{\parsep}{0in} \setlength{\parskip}{0in}
      \setlength{\topsep}{0in} \setlength{\partopsep}{0in} 
      \setlength{\leftmargin}{0.2in}}}{\end{list}}

\begin{document}

\name{Steven Boada\vspace*{.1in}}

\begin{resume}
\section{\sc Contact Information}
\vspace{.05in}
\begin{tabular}{@{}p{3in}p{3in}}
Department of Physics and Astronomy & {\it Phone:}  +1 (615) 200-0119 \\            
4242 TAMU   & {\it E-mail:}  boada@physics.tamu.edu \\         
Texas A\&M University & {\it WWW:} http://boada.github.io \\       
College Station, Texas 77843-4242  & \\     
\end{tabular}

\section{\sc Research Interests}
Observation Cosmology, Large-area Sky Surveys (e.g., DES, LSST, SDSS, ACT, SPT), Galaxy Clusters, High Performance Computing (HPC), Galaxy Evolution, Interacting Galaxies and Morphology.
\vspace*{-3mm}

\section{\sc Education}
\textbf{Texas A\&M University}, College Station, Texas USA\\
%{\em Department of Physics and Astronomy} 
\vspace*{-.1in}
\begin{list1}
    \item[]Ph.D. Candidate, Physics (Astronomy focus; expected graduation date: May 2016)
    \begin{list2}
        %\vspace*{-.15in}
        \item Dissertation Title: ``Galaxy Cluster Dynamics in the Era of Large Spectroscopic\\ Surveys'' 
        \item Advisor: Dr. Casey Papovich
    \end{list2}
\end{list1}
\vspace*{-3mm}

\textbf{The University of Tennessee}, Knoxville, Tennessee USA\\
%{\em Department of Physics and Astronomy}
\vspace*{-.1in}
\begin{list1}
    \item[] M.S., Physics (Computational Astrophysics),  August, 2009
    \begin{list2}
        \vspace*{.05in}
        \item Thesis Title: ``An Automated Approach to the Study and Classification of Colliding and Interacting Galaxies''
        \item Advisor: Dr. Michael Guidry
    \end{list2}
\end{list1}
\vspace*{-3mm}

{\bf The University of Tennessee}, Knoxville, Tennessee USA\\
%{\em Department of Physics and Astronomy} 
\vspace*{-.1in}
\begin{list1}
    \item[] B.S., Physics,  May, 2007
\end{list1}
\vspace*{-3mm}

\section{\sc Professional Experience}
\textbf{Texas A\&M University}, College Station, Texas USA
\vspace{-3mm}

{\em Research Assistant} \hfill \textbf{August, 2010 - Present}
\vspace*{-1mm}

\textbf{The University of Tennessee}, Knoxville, Tennessee USA
\vspace{-3mm}

{\em Research Assistant} \hfill \textbf{August, 2007 - 2009}
\vspace*{-1mm}

\textbf{National Center for Computational Science}, Oak Ridge National Laboratory,
Oak Ridge, Tennessee USA
\vspace*{-4mm}

{\em Visiting Scientist} \hfill \textbf{May, 2007 - August, 2009}\\
Carried out the computing projects required to complete Master's, including
modeling of interacting galaxy systems, machine learning, and other HPC tasks. 
\vspace*{-3mm}

\section{\sc Observing Experience}
Proposals
    \begin{list2}
        \vspace*{.05in}
    \item \emph{Measuring the Masses of X-ray-Selected, Low-Mass Galaxy Clusters and Groups with Integral Field Spectroscopy}\\
		Co-I (PI: N. Mehrtens), McDonald Observatory, 4 nights awarded, 2013
    \item \emph{Measuring the Masses of Galaxy Clusters with Integral Field Spectroscopy}\\
		Co-I (PI: C. Papovich), McDonald Observatory, 9 nights awarded, 2012
    \item \emph{Measuring the Masses of Galaxy Clusters with Integral Field Spectroscopy}\\
		Co-I (PI: C. Papovich), McDonald Observatory, 5 nights awarded, 2012
	\end{list2}
Telescopes
    \begin{list2}
        \vspace*{.05in}
    \item Harlan J. Smith 2.7m Telescope, Mitchell Spectrograph (formerly VIRUS-P), 20+ nights
	\end{list2}
Data Experience
    \begin{list2}
        \vspace*{.05in}
		\item Integral Field Spectroscopy
    	\item Hubble Space Telescope Imaging
		\item Sloan Digital Sky Survey Imaging and Spectroscopy
	\end{list2}

\section{\sc Computing Experience} 
Extensive experience in the processing and application of large astronomical data sets, including: the acquisition and reduction of optical integral field unit spectroscopy, querying large astronomical databases such as the Sloan Digital Sky Survey and the Millennium Simulation, analysis of multi-wavelength imaging from the Hubble Space Telescope. Key computing skills include: mastery of the Python language, and the interface with other languages and tools, considerable experience with large multiprocessor applications (e.g. Gadget-2) and high performance computing systems, supervised and unsupervised machine learning and optimization, GPGPU computing, and participation in open source and collaborative development environments, including version control. Contributor to {\sc AstroPy}. Co-author of {\sc astLib} python library, see http://astlib.sourceforge.net
\vspace*{-3mm}

\section{\sc Teaching and Outreach}
\textbf{Texas A\&M University}, College Station, Texas USA
\vspace{-3mm}

\emph{Teaching Assistant}\hfill \textbf{2010 - Spring, 2015}\\
Supervised undergraduate students for weekly lab sessions, tutoring sessions, grading of homework and quizzes for Basic Astronomy, Overview of Modern Astronomy, and Survey of Astronomy.
\vspace*{-1mm}

\emph{Physics Festival} \hfill \textbf{2010 - Present}\\
Demonstrated physics and astronomy principles for students from elementary through high school and the general public.
\vspace*{-1mm}

\emph{Star Parties} \hfill \textbf{2010 - Present}\\
Discussed astronomical topics and operated telescopes for college students and the general public.
\vspace*{-1mm}

%\pagebreak
\textbf{Nashville State Community College}, Nashville, Tennessee USA
\vspace{-3mm}

\emph{Adjunct Faculty} \hfill \textbf{Spring, 2010}\\
Primary instructor for introductory physics course, Conceptual Physics.
\vspace*{-1mm}

\textbf{The University of Tennessee}, Knoxville, Tennessee USA
\vspace{-3mm}

\emph{Teaching Assistant} \hfill \textbf{August, 2007 - 2009}\\
Supervised laboratory experiences for undergraduate students in Introduction to Modern Physics, and Electricity and Magnetism for Engineering. Designed and taught laboratories for undergraduate Honors Astronomy.
\vspace*{-3mm}

\section{\sc Academic Honors and Awards} 
The University of Tennessee: graduated Magna Cum Laude, Phi Beta Kappa, Sigma
Pi Sigma, President, Society of Physics Students 2006 thru 2007
%\vspace*{-1mm}
\newpage
\section{\sc Grants and Awards}
\begin{list2}
    \item \emph{The Road to the Virgo Cluster: The DECam/IRAC Galaxy Environment Survey}\\
	Co-I (PI: C. Papovich), NSF Alliances for Graduate Education and the Professoriate, 2015
	\item \emph{Graduate Student Presentation Grant}\\
	PI, Texas A\&M University Office of Graduate and Professional Studies, 2015
	\item \emph{Graduate Student Travel Grant}\\
	PI, Texas A\&M University Department of Physics and Astronomy, 2015
\end{list2}

\section{\sc Posters and Presentations}
Talk: 227th AAS Meeting, Kissimmee, FL January, 2016\\
Talk: CANDELS Team Meeting, University of Santa Cruz, Santa Cruz, CA July, 2015\\
Talk: CANDELS Team Meeting, STScI, Baltimore, MD July, 2014\\
Poster: Bashfest Symposium, University of Texas, Austin, TX October, 2013\\
Talk: CANDELS Team Meeting, University of Kentucky, Lexington, KY August, 2013\\
Poster: GMT Science Meeting, University of Chicago, Chicago, IL June, 2013\\
Talk: CANDELS Team Meeting, University of Santa Cruz, Santa Cruz, CA September, 2012\\
Poster: 219th AAS Meeting, Austin, TX January, 2012\\ 
Poster: Bashfest Symposium, University of Texas, Austin, TX October, 2011\\
Talk: Texas A\&M Astronomy Symposium, Texas A\&M University, College Station, TX August, 2011--15\\

\section{\sc References}
\begin{tabular}{@{}p{2in}p{2in}p{2in}}
		Dr. Casey Papovich & Dr. Vithal Tilvi & Dr. Nicholas Suntzeff \\
		Dept. of Physics \& Astronomy & School of Earth \& Space Exploration & Dept. of Physics \& Astronomy \\
		4242 TAMU & P.O. Box 871404 & 4242 TAMU \\
		Texas A\&M University & Arizona State University & Texas A\&M University\\
		College Station, Texas 77843 & Tempe, Arizona 85287 & College Station, Texas 77843 \\
		papovich@physics.tamu.edu & tilvi@asu.edu & nsuntzeff@tamu.edu   
\end{tabular}

%\vspace*{-.25in}  
\end{resume}
\end{document}




